\cleardoublepage
\subsection{Probabililty Distributions and Rules}
\textbf{Learning Goal:} Reason from probability distributions, using probability rules, to answer probability questions.

At this point, you can jump right in and start working basic probability problems.

\textbf{Group Work.}
\begin{enumerate}
\item If the probability for a head for a biased coin is 60\%, what is the probability of a tail? How do you know?\\[.5in]
\item For the biased coin in the previous problem, which is more likely, head or tail?\\[.5in]
\item Surprisingly, the probability of having a boy differs slightly by latitude. People in northern latitudes are more likely to have boys. Here is the distribution of boys in 3-child families for a northern latitude.

\begin{table}[H]\centering
\begin{tabular}{|c|c|c|c|c|} \hline
Boys & 0 & 1 & 2 & 3 \\
\hline
Probability&0.07& 0.41 & 0.34 & 0.18 \\
\hline
\end{tabular}
\end{table}

Show your work or explain your process for each question.
\begin{enumerate}
\item If you choose a 3-child family at random, which is more likely, no boys or 3 boys?\\[.5in]
\item If you choose a 3-child family at random, what is the chance of selecting a family with at least one boy?\\[.5in]
\item If you choose a 3-child family at random, what is the probability of choosing a family with 1 or 2 boys?\\[.5in]
\end{enumerate}
\item The number of eggs in a bird nest has been reported by a birder in the table below

\begin{table}[H]\centering
\begin{tabular}{|c|c|c|c|c|} \hline
Eggs & 0 & 1 & 2 & 3 \\
\hline
 Probability&0.3& 0.4 & 0.3 & 0.2 \\
\hline
\end{tabular}
\end{table}

How do you know something is wrong with the probability calculations?\\[.5in]

\item Complete the contingency table for the data collected from a group of students.
\begin{figure}[H]
\centering{}\includegraphics[width=5in]{./img/BikeCar} 
\end{figure}

\begin{enumerate}
\item If you select a student at random, what is the probability that the student owns a car?\\[.5cm]
\item If you select a student at random, what is the probability that the student owns a car and a bike?\\[.5cm] 
\item If you select a student at random, what is the probability that the student owns a car or a bike?\\[.5cm]
\item If you select a student at random, what is the probability that the student does not own a car and does not own a bike?\\[.5cm] 
\item If you select a student at random, what is the probability that the student does not own a car or a bike?\\[.5cm] 

\end{enumerate}
\end{enumerate}
\textbf{Did I Get This?} In working through the preceding problems you may have found yourself using the same techniques several times. Let's summarize these basic techniques or rules\dots 
\begin{enumerate}
\item The probability of an event must fall between 0 and 1 (or be equal to 0 or 1).
\item When we add the probabilities of all the different outcomes, the probabilities must add to 1.00 (=100\%).
\item If two events are \emph{disjoint} you can add their probabilities.
\end{enumerate}

Find an example of each of the rules in the Group Work problems. Write down the problem number and the corresponding rule.
