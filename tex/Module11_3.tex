\cleardoublepage
\subsection{Conditional Probability and Independence}
\textbf{Learning Goal:} Use conditional probability to identify independent events.

\textbf{Warmup.} The partially completed contingency table for a group of students and how they get to school is given below. The students are categorized by gender.

\begin{table}[h]
\centering
\begin{tabular}{|r|c|c|c|} \hline
&Male&Female&Total\\
\hline
Drive & 30 & 20 & \\
\hline
Walk & 10 & 40 & \\
\hline
Total & & & \\
\hline
\end{tabular}
\end{table}

\begin{enumerate}
\item Complete the contingency table by filling in the marginal totals and grand total.
\item If we randomly select a student, what is the probability that a student walks to school? \\[.5cm]
\item If we randomly select a male student, what is the probability that the student walks to school?\\[.5cm]
\item Referring to the previous two problems, does the additional information---the student is a male---make a difference in the resulting probability?\\[.5cm]
\end{enumerate}

When additional information, in this case the additional knowledge that the student is a male, makes a difference in the estimated probability we say the event and the additional information are \emph{dependent}. Otherwise, we say they are \emph{independent}.

\textbf{Group Work.} 
\begin{enumerate}
\item Again we look at 236 students and where they sit in class. 
\begin{table}[h]
\centering
\begin{tabular}{|c|c|c|c|c|}\hline
&\multicolumn{3}{|c|}{Where do you sit in class?}&\\
\hline
&Front&Middle&Back&Total\\
\hline
Female&37&91&22&150\\
\hline
Male&15&46&25&86\\
\hline
Total&52&137&47&236\\
\hline
\end{tabular}
\end{table}

\begin{enumerate}
\item If we randomly select a student, what is the probability that a student sits in the front of the class? Express your answer as a decimal. \\[.4cm]
\item If we randomly select a female student, what is the probability that she sits in the front of the class? Express your answer as a decimal.\\[.4cm]
\item Is sitting in the front of the class independent of being female?\\[.4cm]
\end{enumerate}
\newpage
\item The 320 beginning STEM students at a technical college are allowed to take one techical class this semester. The table below show their current enrollment by gender.

\begin{table}[h]
\centering
\begin{tabular}{|c|c|c|c|c|}\hline
&\multicolumn{3}{|c|}{Currently Enrolled Course}&\\
\hline
&Engineering 10&Math 100&Physics 101&Total\\
\hline
Female&40&20&20&80\\
\hline
Male&120&60&60&240\\
\hline
Total&160&80&80&320\\
\hline
\end{tabular}
\end{table}

\begin{enumerate}
\item Selecting a student at random, what is the probability that the student is enrolled in Physics 101?\\[.5cm]
\item If we randomly select a male student, what is the probability that he is enrolled in Physics 101? \\[.5cm]
\item Is enrolling in Physics 101 independent of a student's gender? Explain.\\[.5cm]
\end{enumerate}
\item Is having a tattoo independent of whether someone smokes? Use this hypothetical data to answer this question.
\begin{figure}[H]
\centering{}\includegraphics[width=4in]{./img/TattooSmoker} 
\end{figure}
\vspace{.5in}
\item Is buying a hybrid car independent of whether the person exercises regularly? Use this hypothetical data to answer this question.
\begin{figure}[H]
\centering{}\includegraphics[width=4in]{./img/ExerciseHybrid} 
\end{figure}

\end{enumerate}

