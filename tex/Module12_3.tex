\cleardoublepage
\subsection{Finding Areas Under the Normal Curve with Technology}
\textbf{Learning Goal:} Use a normal probability distribution to estimate probabilities and identify unusual events.

\textbf{Specific Learning Objective:} Use StatCrunch and the OLI applet to find areas under the normal curve in order to estimate probabilities.

\textbf{Introduction.} Previously, we used the normal model to estimate probabilities.

For each of the problems below,
\begin{itemize}
\item Label the axis for the normal model with the following values: the mean, mean $\pm$ SD, mean $\pm$ 2 SD, mean $\pm$ 3 SD.
\item Shade the area that represents the answer.
\item Find the probability using the Empirical Rule if you can.
\end{itemize}

Women's heights can be modeled by a normal curve with mean of 64 inches and standard deviation of 4 inches.
\begin{enumerate}
\item Women with heights between 64-inches (5'4'') and 68-inches (5'8'') can wear a medium Leggs pantyhose. What is the probability that a randomly selected woman can wear a medium? 
\begin{figure}[h]
\centering{}\includegraphics[width=3in]{./img/normal} 
\end{figure}
\vspace{.5in}
\item Women with heights between 61 inches (5'1'') and 74 inches (6'2'') fit comfortably in a new fuel-efficient car. What is the probability that a randomly selected woman is comfortable in this car? 
\begin{figure}[h]
\centering{}\includegraphics[width=3in]{./img/normal} 
\end{figure}
\vspace{.5in}

\end{enumerate} 
Now we will use the Normal Calculator in StatCrunch to work both of these problems. Your instructor will demonstrate how to do this. 
\newpage
\textbf{Practice with StatCrunch.} 
\begin{enumerate}
\item Most tests that gauge one's intelligence quotient (IQ) are designed to have a mean of 100 and a standard deviation of 15 and are standardized to fit a normal curve. There is no universal agreement on what IQ constitutes a ``genius'', though in 1916, psychologist Lewis M. Thurman set a guideline of 140 (scaled to 136 in today's tests) for ``potential genius.'' What is the probability that a randomly selected person today is a ``potential genius?''
\begin{enumerate}
\item Label the normal curve and sketch the area that represents the answer.\\[.5in]
\begin{figure}[H]
\centering{}\includegraphics[width=3in]{./img/normal} 
\end{figure}
\item Use the StatCrunch Normal Calculator to find the probability.
\end{enumerate}
\vspace{.5in}
\item Weights of 1-year-old boys are approximately normally distributed, with a mean of 22.8 lbs and a standard deviation of about 2.15 lbs. (Source: About.com) What is the probability that a one-year-old boy weighs at least 20 pounds? 
\begin{enumerate}
\item Label the normal curve and sketch the area that represents the answer.\\[.5in]
\begin{figure}[H]
\centering{}\includegraphics[width=3in]{./img/normal} 
\end{figure}
\item Use the StatCrunch Normal Calculator to find the probability.
\end{enumerate}
\vspace{.5in}

\item Are 1-year-old boys who weigh more than 28 pounds unusual? Why or why not? (Use the information about the weight distribution given in the second problem.)\\[1in]
\item One-year-old boys with weights in the 95th percentile are considered obese. Tom's son weighs 26 pounds. Is he obese by this definition? How do you know? (Use the information about the weight distribution given in the second problem.)\\[1in]
\end{enumerate}

\textbf{Another approach: Using a z-score and the Standard Normal Curve}

The z-score for a value just measures how far the value is from the mean, using the standard deviation as a yardstick. Positive z-scores are above the mean, negative z-scores are below the mean.

\[z=\displaystyle\frac{\mbox{value}-\mbox{mean}}{\mbox{standard deviation}}\]

Some normal calculators, for example the OLI normal calculator, only work with z-scores, and make you convert your value to a z-score in order to get the answer.

Example: Suppose you took a test and scored an 85 out of 100. Suppose that the distribution of test scores for your class had a mean score of 75 with a standard deviation of 10. Your friend took a test and scored a 16 out of 20. Suppose the distribution of test scores for her class had a mean of 10 with a standard deviation of 5. Who performed better relative to their class?

We can't compare the two scores directly because the scales are different. One test had a possible 100 points. The other test had a possible 20 points. We could use percentages or we could use z-scores to compare the two scores.
 
If we use percentages, you scored an 85\%. Your friend scored a 75\%. So you did better.

If we use z-scores, your z-score is (85-75)/10 = 1. Your friend's z-score is (16-10)/5 = 1.2. So relative to the class that the student is in, your friend scored a little higher. Her score is 1.2 standard deviations above the mean for her class, while your score is 1 standard deviation above the mean for your class.

\textbf{Practice with z-scores and the Standard Normal Curve.}
\begin{enumerate}
\item Weights of 1-year-old boys are approximately normally distributed, with a mean of 22.8 lbs and a standard deviation of about 2.15. (Source: About.com) Children whose weights are more than 2 standard deviations above or below the mean are considered unusual. Maria's 1-year-old son weighs 19 pounds. Is he unusual? \\[1in]
\item Ann's son weighs 18 pounds. 
\begin{enumerate}
\item Verify that the z-score for Ann's son is about -2.23.\\[.5cm]
\item Why is the z-score for Ann's son negative? \\[.5cm]
\item What is the probability that a 1-year-old boy weighs 18 pounds or less according to the OLI Normal Calculator?  
\begin{figure}[h]
\centering{}\includegraphics[width=3in]{./img/OLIshot} 
\end{figure}
\vspace{.5in}

\item Compare the probability to the StatCrunch Normal Calculator. Do both calculators give similar probability estimates.
\begin{figure}[h]
\centering{}\includegraphics[width=3in]{./img/SCshot} 
\end{figure}
\vspace{.5in}
\item Are 1-year-old boys in this weight range unusual? Why or why not? \\[.5cm]
\item A 1-year-old boy in the 5th percentile is considered underweight. Is Ann's son considered underweight? How do you know? 
\end{enumerate}

\end{enumerate}


