%%%%%%%%%%%%%%%% Unit 3 Project %%%%%%%%%%%%%

%\textbf{\large{}Unit 3 Project}{\large \par}

\textbf{Instructions:} In your group, chose one of the four options
described below. Analyze the data using what you have learned in Unit
3. Each group will make a poster and present their analysis in a gallery
walk. Based on feedback from the gallery walk, each group will revise
their work and present an improved analysis to the class. Your instructor
may also require each group member to write an analysis and submit
it individually. \\


\textbf{Poster Instructions:} Your poster will include the following: 
\begin{itemize}
\item A statement of the research question 
\item A description of the source of the data 
\item A description of the variables used in the analysis 
\item Scatterplots with clear labels that facilitate easy visual comparison 
\item Explanations that reflect the use of Unit 3 concepts (form, direction,
strength, explanatory and response variables, regression equation,
prediction, and r.
\item An answer to the question based on your analysis of the data \clearpage{}
\end{itemize}
\textbf{\large{}Research Question 1:}\\


\textbf{Introduction:} In the Movie data set, we have movie ratings
from a variety of websites, such as\\ \emph{www.rottentomatoes.com},
\emph{www.imdb.com}, and metascore ratings from \emph{www.metacritic.com}.
\\


\textbf{Research questions (answer both): } 
\begin{itemize}
\item Which is a better predictor of Rotten Tomatoes movie ratings: IMDb
ratings or Metascore ratings? \ 
\item For the variable that correlates with Rotten Tomatoes ratings more
strongly, what is the predicted Rotten Tomatoes rating for a movie
that has an IMDb rating of 7 and a Metascore of 80? Use Unit~3 concepts
to comment on the accuracy of your prediction. 
\end{itemize}
Investigate this question using the data set \emph{Movies.txt}. Support
your answer using concepts from Unit~3. Follow the instructions for
creating a poster.\bigskip{}
\bigskip{}
\bigskip{}


\textbf{\large{}Research Question 2:}{\large \par}

\textbf{Introduction:} Pediatricians use a child's measurements to
predict his or her height and weight as an adult. Researchers studying
child development measured a sample of 136 children at ages 2, 9,
and 18. These children were born in 1928--29 in Berkeley CA. We will
examine the relationships between body measurements for children and
the corresponding measurement for an 18-year-old.\\


Use the data in the file titled \emph{Child Development} to answer
one of the following questions.\bigskip{}


\textbf{Research questions (choose one):}
\begin{enumerate}
\item Can we more accurately estimate an 18-year-old's height or an 18-year-old's
weight using the corresponding measurements for 9-year-olds? 
\item Which is a better predictor of an 18-year-old's height: childhood
height at age 2 or childhood height at age 9? 
\item Is strength at age 9 a good way to predict strength at age 18? For
which gender is strength at age 9 a better predictor of strength at
age 18? 
\end{enumerate}
Support your answer using concepts from Unit 3. \ Follow the instructions
for creating a poster.\bigskip{}
\\
\emph{After answering your research question}, use the stronger relationship
that you found to make a prediction using a regression line. Explain
what you are doing in a way that a parent without statistical training
could understand. Incorporate this into your poster and presentation. 

\textbf{\clearpage{}}\textbf{\large{}Research Question 3:}\\


\textbf{Introduction:} In a course called Introduction to Statistics
at Carnegie Mellon University, a professor wants to gain insight into
his students' performance on the final exam by analyzing exam grades
from earlier in the semester. The data set describes 105 students.
The data is in \emph{gradebook.xls}. Support your answers using concepts
from Unit 3. Follow the instructions for creating a poster.\bigskip{}


\textbf{Research questions (answer both):}
\begin{itemize}
\item In general, which is the best predictor of a student's score on the
final exam, the score on the first midterm or the second midterm?
Use Unit 3 concepts to support your answer. 
\item What is the predicted score on the final exam for a student who scores
a 77 on the first midterm and an 88 on the second midterm? \ 
\end{itemize}
Support your answers using concepts from Unit 3. \ Follow the instructions
for creating a poster.

\bigskip{}


\bigskip{}
\bigskip{}
\bigskip{}
\bigskip{}


\textbf{\large{}Research Question 4:\bigskip{}
}{\large \par}

\textbf{Introduction:} In forensic science, the identification of
a dead body is often based on an incomplete skeleton. Forensic scientists
use ideas from Unit 3 and other statistical techniques to predict
the height and weight of a dead person based on measurements of bones
or body parts. Use the file \emph{Body Measurement.txt} to answer
the following questions.\bigskip{}


\textbf{Research questions (answer both): }
\begin{itemize}
\item In a war zone a foot and an ankle are recovered but no other parts
of the body are found. The ankle girth is 25 centimeters. What is
the predicted height and the predicted weight of this person? (Remember
to always plot the data before making predictions!) 
\item Which is a more accurate prediction? Why do you think so? 
\end{itemize}
Support your answer using concepts from Unit 3. Follow the instructions
for creating a poster.\newpage{}%
\fbox{\begin{minipage}[t]{1\columnwidth}%
\begin{center}
\textbf{\textsc{\large{}Definition Of Variables}}
\par\end{center}%
\end{minipage}}

\textbf{\large{}Option 1: Movies in }\textbf{\emph{\large{}Movies.txt}}{\large \par}

This data set describes 75 movies listed in the top 100 USA box office
sales of all time. Data was taken from IMDb.com in Spring 2014.

\begin{tabbing}
\hspace*{2in}\=\kill
\textbf{\emph{Variable}}\> \textbf{Variable Definition}\\
\emph{Year}\> Year\\
\emph{Studio Name}\> Studio Name\\
\emph{Studio Type}\> Studio Type (Big 6, Other)\\
\emph{Genre }\> Genre (Action/Adventure, Other)\\
\emph{Budget (millions \$)}\> Budgeted Cost to Produce (millions \$)\\
\emph{US Box Office (millions \$) }\> US Box Office Revenues (millions
\$)\\
\emph{First Week End (millions \$)}\> First Week End Gross Box Office
Revenues (millions \$)\\
\emph{Movie\_Length (minutes)}\> Length (minutes)\\
\emph{Trailer\_Length (seconds)}\> Trailer Length (seconds)\\
\emph{Director}\> Name of the Director\\
\emph{Director\_Gender}\> Gender of the Director\\
\emph{Director\_Race}\> Race of the Director\\
\emph{Star}\> Name of the Star\\
\emph{Star\_Gender}\> Gender of the Star\\
\emph{Star\_Race }\> Race of the Star\\
\emph{Costar}\> Name of the Costar\\
\emph{Costar\_Gender }\> Gender of the Costar\\
\emph{Costar\_Race}\> Race of the Costar\\
\emph{IMDb\_Rating}\> How it was rated by IMDb\\
\emph{Metascore}\> How it was rated by Metascore\\
\emph{Metacriticcom\_rating}\> How it was rated by Metacriticcom\\
\emph{Rotten\_Tomatoes}\> How it was rated by Rotten Tomatoes\\
\emph{Number\_of\_Oscars}\> Number of Oscars Won\\
\emph{Oscar\_Nominations}\> Number of Oscar Nominations\\
\emph{Oscar\_Winner}\> Did this movie win an Oscar?\\
\end{tabbing}

\newpage
\textbf{\large{}Option 2: Body measurements in the file}\textbf{\emph{\large{}
Child\_development}}{\large \par}

Data from a study by Tuddenham, R. D., \& Snyder, M. M. 1954. Physical
growth of California boys and girls from birth to eighteen years.
Child Development 1:183-364.

{*}Bodytype rates a person on a scale of slender (1) to fat (7). 

\begin{tabbing}
\hspace*{1.2in}\=\kill
\textbf{\emph{Variable}}\> \textbf{Variable Definition}\\
\emph{CaseNo}\>  Number assigned to child\\
\emph{Gender}\>  Child's gender\\
\emph{Weight\_2}\>  Age 2 weight (kg)\\
\emph{Height\_2}\>  Age 2 height (cm)\\
\emph{Weight\_9}\>  Age 9 weight (kg)\\
\emph{Height\_9}\>  Age 9 height (cm)\\
\emph{Leg\_9}\>Age 9 leg circumference (cm)\\
\emph{Strength\_9}\>  Age 9 strength (kg)\\
\emph{Weight\_18}\>  Age 18 weight (kg)\\
\emph{Height\_18}\>  Age 18 height (cm)\\
\emph{Leg\_18}\> Age 18 leg circumference (cm)\\
\emph{Strength\_18}\>  Age 18 strength (kg)\\
\emph{Bodytype\_18}\>  1-7 scale{*} 
\end{tabbing}

\bigskip
\bigskip
\textbf{\large{}Option 3: Grades in }\textbf{\emph{\large{}gradebook.xls}}{\large \par}

\begin{tabbing}
\hspace*{1.2in}\=\kill
\textbf{\emph{Variable}}\> \textbf{Variable Definition}\\
\emph{Midterm1}\>  Student\textquoteright s score on the first midterm
(0-100 scale) \\
\emph{Midterm2}\>  Student\textquoteright s score on the second midterm
(0-100 scale) \\
\emph{Diff.Mid}\>  The difference between the two midterm exam scores
(midterm1 - midterm2)\\ 
\emph{Extra credit}\>  Did the student turn in the extra credit assignment?
(0=NO, 1=YES) \\
\emph{Final}\>  Student's score on the final (0-100 scale) \\
\emph{Class}\>  Student's class (1=Freshman, 2=Sophomore, 3=Junior, 4=Senior) 
\end{tabbing}
\newpage
\textbf{\large{}Option 4: Body measurements in }\textbf{\emph{\large{}Body
Measurement.txt}}{\large \par}

These data were collected to investigate the correspondence between
frame size, girths, and weight of physically active young men and
women, most of whom were within normal weight range. A goal of this
investigation was to develop predictive techniques to assess the lean/fat
body composition of individuals. The individuals were not randomly
sampled. Measurements were initially taken by Grete Heinz and Louis
J. Peterson at San Jose State University and at the U.S. Naval Postgraduate
School in Monterey, California. Later, measurements were taken at
dozens of California health and fitness clubs by technicians under
the supervision of one of these authors. See article in Journal of
Statistics Education, Volume 11, Number 2 (2003), www.amstat.org/publications/jse/v11n2/datasets.heinz.html.
Exploring Relationships in Body Dimensions, by Grete Heinz, Louis
J. Peterson, and Carter J. Kerk.

\begin{tabbing}
\hspace*{1.2in}\=\kill
\textbf{\emph{Variable}}\> \textbf{Variable Definition}\\
\emph{Gender}\>  Gender (male, female)\\
\emph{Age}\>  Age (Years)\\
\emph{Height}\>  Height (Centimeters)\\
\emph{Weight}\>  Weight (Kilograms)\\
\emph{Pelvic\_dia}\>  Pelvic Diameter (Centimeters)\\
\emph{Chest\_depth}\>  Chest Depth (Centimeters) \\
\emph{Chest\_dia}\>  Chest Diameter (Centimeters)\\
\emph{Elbow\_dia}\>  Elbow Diameter (Centimeters)\\
\emph{Wrist\_dia}\>  Wrist Diameter (Centimeters)\\
\emph{Knee\_dia}\>  Knee Diameter (Centimeters) \\
\emph{Ankle\_dia}\>  Ankle Diameter (Centimeters)\\
\emph{Shoulder\_girth}\>  Shoulder Girth (Centimeters)\\
\emph{Chest\_girth }\> Chest Girth (Centimeters)\\
\emph{Waist\_girth}\>  Waist Girth (Centimeters)\\
\emph{Abdominal\_girth}\>  Abdominal Girth (Centimeters)\\
\emph{Hip\_girth}\>  Hip Girth (Centimeters)\\
\emph{Thigh\_girth}\>  Thigh Girth (Centimeters)\\
\emph{Bicep\_girth}\>  Bicep Girth (Centimeters)\\
\emph{Forearm\_girth}\>  Forearm Girth (Centimeters)\\
\emph{Knee\_girth}\>  Knee Girth (Centimeters)\\
\emph{Calf\_girth}\>  Calf Girth (Centimeters)\\
\emph{Ankle\_girth}\>  Ankle Girth (Centimeters)\\
\emph{Wrist\_girth}\>  Wrist Girth (Centimeters)
\end{tabbing}
