%%%%%%%%%%%%%%%% Unit 4 Project %%%%%%%%%%%%%
\textbf{Instructions:} In your group, choose one of the five options described below. Analyze the data using what you have learned in Unit 4. Each group will make a poster and present their analysis in a gallery walk. Based on feedback from the gallery walk, each group will revise their work and present an improved analysis to the class. Your instructor may also require each group member to write an analysis and submit it individually.\\ 


\textbf{Poster Preparation Instructions:} Your poster will include the following: 
\begin{itemize}
\item A statement of the research question 
\item A description of the source of the data 
\item A description of the variables used in the analysis 
\item A contingency table (two-way table)
\item Calculations of relevant percentages
\item Pie charts or bar charts to support the analysis
\item An answer to the question based on your analysis of the data
\end{itemize}

\textbf{Option 1: Cheating}\\
Research questions \textbf{(do both)}:
\begin{itemize}
\item Are college students willing to report cheating?
\item Is the willingness to report cheating related to gender?
\end{itemize}

Investigate these questions for the students described in \emph{body\_image.xls}. 

This data comes from a survey of university students at Carnegie Mellon University in Pittsburgh, PA.

Here are the survey questions and associated variables: 
\begin{tabbing}
\hspace*{3in}\=\kill
\emph{Are you a male or a female?}\> Gender (male, female)\\
\emph{What is your height in inches?}\> Height (in inches)\\
\emph{What is your GPA?}\> GPA\\
\emph{What was your high school GPA?}\> HS GPA\\
\emph{Where do you tend to sit in class?}\> Seat (F=front, M=middle, B=back)\\
\emph{How do you feel about your weight?}\> WtFeel (OverWt, AboutRt, UnderWt)\\
\emph{Would you report cheating if you witnessed it?}\> (yes, no)
\end{tabbing}

\newpage
\textbf{Option 2: Gender and Body Image} 

\textbf{Research question:} Do female college students tend to feel differently about their weight compared to male college students? 

Investigate this question for the students described in \emph{body\_image.xls}. 

This data comes from a survey of university students at Carnegie Mellon University in Pittsburgh, PA.

Here are the survey questions and associated variables: 
\begin{tabbing}
\hspace*{3.5in}\=\kill\
\emph{Are you a male or a female?}\> Gender (male, female)\\
\emph{What is your height in inches?}\> Height (in inches)\\
\emph{What is your GPA?}\> GPA\\
\emph{What was your high school GPA?}\> HS GPA\\
\emph{Where do you tend to sit in class?}\> Seat (F=front, M=middle, B=back)\\
\emph{How do you feel about your weight?}\> WtFeel (OverWt, AboutRt, UnderWt)\\
\emph{Would you report cheating if you witnessed it?}\> (yes, no)
\end{tabbing}
\bigskip
\textbf{Option 3: Breakfast cereals} 

\textbf{Research question:} Are cereals being intentionally marketed to children?

Investigate this question for the cereals described in \emph{Cereals.txt}. 

This data describes 75 breakfast cereals. A researcher collected this data from a popular grocery store chain.

Here is a description of the variables: 
\begin{tabbing}
\hspace*{1.2in}\=\kill
\emph{Name}\>  Name of cereal \\
\emph{Manufacturer}\>  Manufacturer of cereal\\
\emph{Type}\>  Cereal type (hot or cold)\\
\emph{Shelf}\>  Display shelf at the grocery store\\
\emph{Target}\> Target audience for cereal (Child or Adult)\\
\emph{Calories}\>  Calories per serving\\
\emph{Cups}\>  Number of cups in one serving\\
\emph{Weight}\>  Weight in ounces of one serving\\
\emph{Protein}\>  Grams of protein in one serving\\
\emph{Fat}\>  Grams of fat in one serving\\
\emph{Sodium}\>  Milligrams of sodium in one serving\\
\emph{Fiber}\>  Grams of dietary fiber in one serving\\
\emph{Carbs}\>  Grams of complex carbohydrates in one serving\\
\emph{Sugars}\>  Grams of sugars in one serving\\
\emph{Potassium}\>  Milligrams of potassium in one serving\\
\emph{Vitamins}\>  Vitamins and minerals - 0, 25, or 100\% of daily need in one serving\\
\emph{Rating}\>  Consumer Reports overall rating of nutritional value  
\end{tabbing}

\newpage
\textbf{Option 4: Low Birth Weight} 

\textbf{Research question:}Is visiting a doctor during the early stages of pregnancy associated with a lower incidence of low birth weights? 

Investigate this question for the women described in \emph{low\_birth\_weight\_study.txt}. 

This data describes 189 pregnant women who participated in a medical study at a Massachusetts hospital.

Here are the variables:
\begin{tabbing}
\hspace*{1in}\=\kill
\emph{AGE} \> Age of mother (in years)\\
\emph{LWT} \> Weight of mother at the last menstrual period (in pounds)\\
\emph{BWT} \> Birth weight of the baby (in grams)\\
\emph{Low\_Wt} \> whether the baby was born weighing less than 2500 grams (No, Yes)\\
\emph{Smoker} \> Smoking status during pregnancy (No, Yes)\\
\emph{Labor} \> History of premature labor (No, Yes)\\
\emph{Visit} \> Did the mother visit a doctor during the first trimester of pregnancy (No, Yes)
\end{tabbing}
\textbf{Option 5: Movies}


\textbf{Research questions (choose one):}\\
1. Are the Big 6 studios more likely to choose a person of color as a lead star? \\
2. Are women more likely to star in Action/Adventure movies or other types of movies?

Investigate these questions for the movies described in \emph{Movies.txt}. This data set describes 75 movies listed in the top 100 USA box office sales of all time. Data was taken from IMDb.
\begin{tabbing}
\hspace*{1.5in}\=\kill
\emph{Year}\> Year movie was released\\
\emph{Studio}\> Studios categorized as Big6 or Other \\
\emph{Studio Name}\> Name of studio producing the movie\\
\emph{Genre}\> Action/Adventure or Other\\
\emph{Budget}\> Movie budget in millions of dollars\\
\emph{US Box Office}\> total box office sales in millions of dollars\\
\emph{Opening Week}\> box office sales for opening week in millions of dollars\\
\emph{Movie Length}\> length of movie in minutes\\
\emph{Trailer Length}\> length of advertising trailer in seconds\\
\emph{Director}\> Name of the movie's director\\
\emph{Director Gender}\> male or female\\
\emph{Director Race}\> W (white) or POC (person of color)\\
\emph{Star}\> Name of the movie's lead star\\
\emph{Star Gender}\> male or female\\
\emph{Star Race}\> W (white) or POC (person of color) \\
\emph{Costar}\> Name of the movie's main costar\\
\emph{Costar Gender}\> male or female\\
\emph{Costar Race}\> W (white) or POC (person of color)\\
\emph{IMDb\_Rating}\> Average IMDb user rating scale of 1-10\\
\emph{Metascore}\> Score out of 100, based on major critic reviews as provided by Metacritic.com\\
\emph{Metacriticcom\_rating}\> Number of critic reviews used to calculate the Metascore\\
\emph{Rotten\_Tomatoes}\> Score out of 100, based on authors from writing guilds or film critic associations\\
\emph{Number of Oscars}\> number of Oscars won by the movie\\
\emph{Oscar Nominations}\> number of Oscar nominations for the movie\\
\emph{Oscar Winner}\> whether the movie won an Oscar (yes or no)
\end{tabbing}

