\subsection{Introduction}
\textbf{Learning Goal:} Interpret a probabillty as a long run relative frequency of an event.

\textbf{Chances of Heads with a Biased Coin.} Here is the output from a biased coin application.  
\begin{figure}[H]
\centering{}\includegraphics[width=4in]{./img/ShinyBiasedCoin} \caption{Biased Coin Toss, https://lmcstatistics.shinyapps.io/BiasedCoin}
\label{fig:normSDs}
\end{figure}

Using the application, your instructor will lead you through a demonstration that will help answer the following questions.

\begin{description}
\item[a)] What is the probability of heads for this biased coin?
\item[b)] A random event is unpredictable in the short run, but has a pattern in the long run that relates to probability.
\begin{itemize}
\item Explain why flipping a biased coin is a random event.\\[.5in]
\item Explain how we determine the probability of a head from the biased coin flipping demonstration.\\[.5in]
\end{itemize}
\end{description}

\textbf{Warmup.} In the biased coin simulation we used relative frequencies to estimate the probability of getting a head when tossing a biased coin. This is the same strategy we have used when we talked about probabilities with two-way tables. Let's review that now.

Suppose we pick an LMC student at random. Let's estimate the probability they were born in a given month. Because we don't have birthday data for all LMC students, we will use students in our class as a representative sample of all students. Your instructor will guide through completing the table below.

\begin{table}[h]\centering
\begin{tabular}{|c|c|c|c|c|c|c|c|c|c|c|c|c|c|c|}\hline
Month&Jan&Feb&Mar&Apr&May&Jun&Jul&Aug&Sep&Oct&Nov&Dec&TOTAL\\
\hline
Freq.&&&&&&&&&&&&&\\
\hline
Rel. Freq.&&&&&&&&&&&&&\\
\hline
\end{tabular}
\end{table}

\textbf{Constructing the Table.}
\begin{description}
\item[A.] Complete the 2nd line of the table by adding your birthday to the count for the appropriate month.
\item[B.] Add the TOTAL entry for the second line of the table after all students have been counted.
\item[C.] Using the 2nd line, including the total, complete the 3rd line of the table.
\end{description}
\newpage
\textbf{Applying the Table.} Use the relative frequencies (line 3) to answer some questions as a class\dots

\begin{enumerate}
\item Which month is least likely?\\[.5cm]
\item Which month is most likely?\\[.5cm]
\item What is the smallest \emph{possible} entry for line 3? What would that mean in terms of your class?\\[.5cm]
\item What is the total for the third line, and why does this make sense?\\[.5cm]
\item Using our estimated probabilities, if you pick an LMC student at random, what are the chances\dots
\begin{enumerate}
\item They were born in February?\\[.5cm]
\item They were NOT born December?\\[.5cm]
\item They were born over the summer (June through August)?\\[.5cm]
\end{enumerate}
\end{enumerate}

\textbf{Group Work.}
\begin{enumerate}
\item \textbf{Getting to School.} Use the class data to complete this table.

\begin{table}[h]\centering
\begin{tabular}{|c|c|c|}\hline
Method:&Frequency&Relative Frequency\\
\hline
Drive&&\\
\hline
Bus&&\\
\hline
Get a Ride&&\\
\hline
Other&&\\
\hline
TOTAL&&\\
\hline
\end{tabular}
\end{table}
\begin{enumerate}
\item If you pick an LMC student at random, what are the chances she gets a ride to school with someone?\\[.5cm] 
\item What are the chances she rides the bus?\\[.5cm]
\end{enumerate}
\newpage
\item The graph below shows 1000 rolls of a die.

\begin{figure}[H]
\centering{}\includegraphics[width=3in]{./img/DieRoll1000} 
\end{figure}

\begin{enumerate}
\item What is the probability of rolling a 1 with this die?\\[.5cm]
\item What would you expect the graph to look like if the die is fair?\\[.5cm]
\item Is the die fair? Why or why not?\\[.5cm]
\end{enumerate}
\end{enumerate}

\textbf{Wrapup.} Take a few minutes in your group to discuss and answer these questions.
\begin{enumerate}
\item How do you estimate a probability from a table of frequency counts?\\[.5in]
\item Probability is the likelihood of a random event occurring. A random event is an event whose outcome is not predictable in the short run, but there is a pattern in the long run that allows us to predict the likelihood of the event.
\begin{enumerate}
\item Explain why rolling a die is a random event.\\[.5cm]
\item Explain how we could determine the probability of rolling a 1 from this die.
\end{enumerate}
\end{enumerate}

